\documentclass{article}

\usepackage[utf8]{inputenc}

\usepackage{amsmath}

\usepackage[spanish]{babel}

\title{Apuntes de programación lineal}

\author{María y Emmanuel}

\begin{document}

\maketitle

\tableofcontents

\section{Introducción}

La forma estándar de un problema de programación lineal es:
Dada una matriz $A$ y vectores $b,c$, maximizar $c^Tx$ sujeto a
$Ax\leq b$.

La forma simplex de un problema de programación lineal es:
Dada una matriz $A$ y vectores $b,c$, maximizar $c^Tx$ sujeto a
$Ax= b$.

\section{Tabla}

Cómo colocar una tabla en el documento.

\bigskip

\begin{tabular}{|c|c|c|}
  \hline
  &A&B\\
  \hline
  Maquina 1&1&2\\
  Maquina 2&1&1\\
  \hline
  
\end{tabular}

\section{Matriz}

Cómo colocar una matriz.

\begin{equation}
  \label{eq:1}
 AB=\begin{pmatrix}
    0&6&2\\
    3&7&9
  \end{pmatrix}
  \begin{pmatrix}
    2&7\\
    8&3\\
    0&13
  \end{pmatrix}  
\end{equation}

\section{Ejemplo}

Un gerente está planeando cómo distribuir la producción de dos productos entre dos máquinas. Para ser manufacturado cada producto requiere cierto tiempo (en horas) en cada una de las máquinas.
El tiempo requerido está resumido en la siguiente tabla:

\bigskip

\begin{tabular}{|c|c|c|}
  \hline
  &A&B\\
  \hline
  Maquina 1&1&2\\
  Maquina 2&1&1\\
  \hline
  
\end{tabular}

\bigskip

La máquina 1 está disponible 40 horas a la semana y la 2 está
disponible 34 horas a la semana. Si la utilidad obtenida al vender los
productos A y B es de 2, 3 pesos por unidad, respectivamente, ¿cuál
debe ser la producción semanal que maximiza la utilidad? ¿Cuál es la utilidad máxima?

\section{Solución}

Se tiene que maximizar 



\end{document}
